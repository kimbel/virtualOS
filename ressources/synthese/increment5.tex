%!TEX root = synthese.tex
\newpage
\section{Design Pattern - Command - suite}

\subsection{Objectifs}

Dans cet incrément il s’agit de créer une série de commande en plus des commandes cat et ls déjà existantes.  Nous devons donc créer les commandes :\\
\begin{itemize}
\item RM : Supprimer un dossier, un fichier ou un lien;
\item MKDIR : Créer un dossier;
\item TOUCH : Créer un fichier;
\item LN : Créer un lien.\\
\end{itemize}

Pour cet incrément, nous réutiliserons le pattern Commande et Visiteur.

\subsection{Implémentation}

Le diagramme UML de cet incrément est identique à celui de l'incrément 4.\\

Nous avons créé les classes \emph{Command} et \emph{Visitor} associées.

\begin{lstlisting}
public class CommandTOUCH implements Command {...}
public class VisitorTOUCH implements VisitorNode {...}
\end{lstlisting}

Pour éviter de surcharger cette synthèse de code, je vous laisse voir les implémentations de ces commandes sur notre lien Github : \url{https://github.com/percenuage/virtualOS.git}

Nous avons utilisé une énumération afin de lister les commandes : \emph{CommandEnum}.

\begin{lstlisting}
public enum CommandEnum {

    LS ("ls"), CAT ("cat"), MKDIR ("mkdir"), 
    RM ("rm"), TOUCH ("touch"), LN ("ln"),
    NOT_FOUND ("commande introuvable");

    private String command;

    private CommandEnum(String command) {
        this.command = command;
    }

    public String getCommand() {
        return command;
    }

    public static CommandEnum getEnum(String command) {
        if (command != null) {
            for (CommandEnum value : values()) {
                if (value.getCommand().equals(command)) {
                    return value;
                }
            }
        }
        return  NOT_FOUND;
    }
}
\end{lstlisting}

On a également utilisé un switch sur l'enumération dans la méthode \emph{executeCommands()} dans la classe \emph{TerminalOS} afin de rediriger les exécutions des commandes. Le \emph{executeCommands} de la classe \emph{User} a également été modifiée (cf. github).\\

Nous n'avons pas oublié de déclarer et d'intancier ces nouvelles commandes dans \emph{TerminalOS} et \emph{ExplorerOS}.

